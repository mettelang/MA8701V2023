% Options for packages loaded elsewhere
\PassOptionsToPackage{unicode}{hyperref}
\PassOptionsToPackage{hyphens}{url}
\PassOptionsToPackage{dvipsnames,svgnames,x11names}{xcolor}
%
\documentclass[
  letterpaper,
  DIV=11,
  numbers=noendperiod]{scrartcl}

\usepackage{amsmath,amssymb}
\usepackage{lmodern}
\usepackage{iftex}
\ifPDFTeX
  \usepackage[T1]{fontenc}
  \usepackage[utf8]{inputenc}
  \usepackage{textcomp} % provide euro and other symbols
\else % if luatex or xetex
  \usepackage{unicode-math}
  \defaultfontfeatures{Scale=MatchLowercase}
  \defaultfontfeatures[\rmfamily]{Ligatures=TeX,Scale=1}
\fi
% Use upquote if available, for straight quotes in verbatim environments
\IfFileExists{upquote.sty}{\usepackage{upquote}}{}
\IfFileExists{microtype.sty}{% use microtype if available
  \usepackage[]{microtype}
  \UseMicrotypeSet[protrusion]{basicmath} % disable protrusion for tt fonts
}{}
\makeatletter
\@ifundefined{KOMAClassName}{% if non-KOMA class
  \IfFileExists{parskip.sty}{%
    \usepackage{parskip}
  }{% else
    \setlength{\parindent}{0pt}
    \setlength{\parskip}{6pt plus 2pt minus 1pt}}
}{% if KOMA class
  \KOMAoptions{parskip=half}}
\makeatother
\usepackage{xcolor}
\setlength{\emergencystretch}{3em} % prevent overfull lines
\setcounter{secnumdepth}{5}
% Make \paragraph and \subparagraph free-standing
\ifx\paragraph\undefined\else
  \let\oldparagraph\paragraph
  \renewcommand{\paragraph}[1]{\oldparagraph{#1}\mbox{}}
\fi
\ifx\subparagraph\undefined\else
  \let\oldsubparagraph\subparagraph
  \renewcommand{\subparagraph}[1]{\oldsubparagraph{#1}\mbox{}}
\fi

\usepackage{color}
\usepackage{fancyvrb}
\newcommand{\VerbBar}{|}
\newcommand{\VERB}{\Verb[commandchars=\\\{\}]}
\DefineVerbatimEnvironment{Highlighting}{Verbatim}{commandchars=\\\{\}}
% Add ',fontsize=\small' for more characters per line
\usepackage{framed}
\definecolor{shadecolor}{RGB}{241,243,245}
\newenvironment{Shaded}{\begin{snugshade}}{\end{snugshade}}
\newcommand{\AlertTok}[1]{\textcolor[rgb]{0.68,0.00,0.00}{#1}}
\newcommand{\AnnotationTok}[1]{\textcolor[rgb]{0.37,0.37,0.37}{#1}}
\newcommand{\AttributeTok}[1]{\textcolor[rgb]{0.40,0.45,0.13}{#1}}
\newcommand{\BaseNTok}[1]{\textcolor[rgb]{0.68,0.00,0.00}{#1}}
\newcommand{\BuiltInTok}[1]{\textcolor[rgb]{0.00,0.23,0.31}{#1}}
\newcommand{\CharTok}[1]{\textcolor[rgb]{0.13,0.47,0.30}{#1}}
\newcommand{\CommentTok}[1]{\textcolor[rgb]{0.37,0.37,0.37}{#1}}
\newcommand{\CommentVarTok}[1]{\textcolor[rgb]{0.37,0.37,0.37}{\textit{#1}}}
\newcommand{\ConstantTok}[1]{\textcolor[rgb]{0.56,0.35,0.01}{#1}}
\newcommand{\ControlFlowTok}[1]{\textcolor[rgb]{0.00,0.23,0.31}{#1}}
\newcommand{\DataTypeTok}[1]{\textcolor[rgb]{0.68,0.00,0.00}{#1}}
\newcommand{\DecValTok}[1]{\textcolor[rgb]{0.68,0.00,0.00}{#1}}
\newcommand{\DocumentationTok}[1]{\textcolor[rgb]{0.37,0.37,0.37}{\textit{#1}}}
\newcommand{\ErrorTok}[1]{\textcolor[rgb]{0.68,0.00,0.00}{#1}}
\newcommand{\ExtensionTok}[1]{\textcolor[rgb]{0.00,0.23,0.31}{#1}}
\newcommand{\FloatTok}[1]{\textcolor[rgb]{0.68,0.00,0.00}{#1}}
\newcommand{\FunctionTok}[1]{\textcolor[rgb]{0.28,0.35,0.67}{#1}}
\newcommand{\ImportTok}[1]{\textcolor[rgb]{0.00,0.46,0.62}{#1}}
\newcommand{\InformationTok}[1]{\textcolor[rgb]{0.37,0.37,0.37}{#1}}
\newcommand{\KeywordTok}[1]{\textcolor[rgb]{0.00,0.23,0.31}{#1}}
\newcommand{\NormalTok}[1]{\textcolor[rgb]{0.00,0.23,0.31}{#1}}
\newcommand{\OperatorTok}[1]{\textcolor[rgb]{0.37,0.37,0.37}{#1}}
\newcommand{\OtherTok}[1]{\textcolor[rgb]{0.00,0.23,0.31}{#1}}
\newcommand{\PreprocessorTok}[1]{\textcolor[rgb]{0.68,0.00,0.00}{#1}}
\newcommand{\RegionMarkerTok}[1]{\textcolor[rgb]{0.00,0.23,0.31}{#1}}
\newcommand{\SpecialCharTok}[1]{\textcolor[rgb]{0.37,0.37,0.37}{#1}}
\newcommand{\SpecialStringTok}[1]{\textcolor[rgb]{0.13,0.47,0.30}{#1}}
\newcommand{\StringTok}[1]{\textcolor[rgb]{0.13,0.47,0.30}{#1}}
\newcommand{\VariableTok}[1]{\textcolor[rgb]{0.07,0.07,0.07}{#1}}
\newcommand{\VerbatimStringTok}[1]{\textcolor[rgb]{0.13,0.47,0.30}{#1}}
\newcommand{\WarningTok}[1]{\textcolor[rgb]{0.37,0.37,0.37}{\textit{#1}}}

\providecommand{\tightlist}{%
  \setlength{\itemsep}{0pt}\setlength{\parskip}{0pt}}\usepackage{longtable,booktabs,array}
\usepackage{calc} % for calculating minipage widths
% Correct order of tables after \paragraph or \subparagraph
\usepackage{etoolbox}
\makeatletter
\patchcmd\longtable{\par}{\if@noskipsec\mbox{}\fi\par}{}{}
\makeatother
% Allow footnotes in longtable head/foot
\IfFileExists{footnotehyper.sty}{\usepackage{footnotehyper}}{\usepackage{footnote}}
\makesavenoteenv{longtable}
\usepackage{graphicx}
\makeatletter
\def\maxwidth{\ifdim\Gin@nat@width>\linewidth\linewidth\else\Gin@nat@width\fi}
\def\maxheight{\ifdim\Gin@nat@height>\textheight\textheight\else\Gin@nat@height\fi}
\makeatother
% Scale images if necessary, so that they will not overflow the page
% margins by default, and it is still possible to overwrite the defaults
% using explicit options in \includegraphics[width, height, ...]{}
\setkeys{Gin}{width=\maxwidth,height=\maxheight,keepaspectratio}
% Set default figure placement to htbp
\makeatletter
\def\fps@figure{htbp}
\makeatother
\newlength{\cslhangindent}
\setlength{\cslhangindent}{1.5em}
\newlength{\csllabelwidth}
\setlength{\csllabelwidth}{3em}
\newlength{\cslentryspacingunit} % times entry-spacing
\setlength{\cslentryspacingunit}{\parskip}
\newenvironment{CSLReferences}[2] % #1 hanging-ident, #2 entry spacing
 {% don't indent paragraphs
  \setlength{\parindent}{0pt}
  % turn on hanging indent if param 1 is 1
  \ifodd #1
  \let\oldpar\par
  \def\par{\hangindent=\cslhangindent\oldpar}
  \fi
  % set entry spacing
  \setlength{\parskip}{#2\cslentryspacingunit}
 }%
 {}
\usepackage{calc}
\newcommand{\CSLBlock}[1]{#1\hfill\break}
\newcommand{\CSLLeftMargin}[1]{\parbox[t]{\csllabelwidth}{#1}}
\newcommand{\CSLRightInline}[1]{\parbox[t]{\linewidth - \csllabelwidth}{#1}\break}
\newcommand{\CSLIndent}[1]{\hspace{\cslhangindent}#1}

\KOMAoption{captions}{tableheading}
\makeatletter
\makeatother
\makeatletter
\makeatother
\makeatletter
\@ifpackageloaded{caption}{}{\usepackage{caption}}
\AtBeginDocument{%
\ifdefined\contentsname
  \renewcommand*\contentsname{Table of contents}
\else
  \newcommand\contentsname{Table of contents}
\fi
\ifdefined\listfigurename
  \renewcommand*\listfigurename{List of Figures}
\else
  \newcommand\listfigurename{List of Figures}
\fi
\ifdefined\listtablename
  \renewcommand*\listtablename{List of Tables}
\else
  \newcommand\listtablename{List of Tables}
\fi
\ifdefined\figurename
  \renewcommand*\figurename{Figure}
\else
  \newcommand\figurename{Figure}
\fi
\ifdefined\tablename
  \renewcommand*\tablename{Table}
\else
  \newcommand\tablename{Table}
\fi
}
\@ifpackageloaded{float}{}{\usepackage{float}}
\floatstyle{ruled}
\@ifundefined{c@chapter}{\newfloat{codelisting}{h}{lop}}{\newfloat{codelisting}{h}{lop}[chapter]}
\floatname{codelisting}{Listing}
\newcommand*\listoflistings{\listof{codelisting}{List of Listings}}
\makeatother
\makeatletter
\@ifpackageloaded{caption}{}{\usepackage{caption}}
\@ifpackageloaded{subcaption}{}{\usepackage{subcaption}}
\makeatother
\makeatletter
\@ifpackageloaded{tcolorbox}{}{\usepackage[many]{tcolorbox}}
\makeatother
\makeatletter
\@ifundefined{shadecolor}{\definecolor{shadecolor}{rgb}{.97, .97, .97}}
\makeatother
\makeatletter
\makeatother
\ifLuaTeX
  \usepackage{selnolig}  % disable illegal ligatures
\fi
\IfFileExists{bookmark.sty}{\usepackage{bookmark}}{\usepackage{hyperref}}
\IfFileExists{xurl.sty}{\usepackage{xurl}}{} % add URL line breaks if available
\urlstyle{same} % disable monospaced font for URLs
\hypersetup{
  pdftitle={MA8701 Advanced methods in statistical inference and learning},
  pdfauthor={Mette Langaas},
  colorlinks=true,
  linkcolor={blue},
  filecolor={Maroon},
  citecolor={Blue},
  urlcolor={Blue},
  pdfcreator={LaTeX via pandoc}}

\title{MA8701 Advanced methods in statistical inference and learning}
\usepackage{etoolbox}
\makeatletter
\providecommand{\subtitle}[1]{% add subtitle to \maketitle
  \apptocmd{\@title}{\par {\large #1 \par}}{}{}
}
\makeatother
\subtitle{L9: Lasso-variants for the linear model}
\author{Mette Langaas}
\date{2/5/23}

\begin{document}
\maketitle
\ifdefined\Shaded\renewenvironment{Shaded}{\begin{tcolorbox}[interior hidden, boxrule=0pt, borderline west={3pt}{0pt}{shadecolor}, enhanced, frame hidden, breakable, sharp corners]}{\end{tcolorbox}}\fi

\renewcommand*\contentsname{Table of contents}
{
\hypersetup{linkcolor=}
\setcounter{tocdepth}{3}
\tableofcontents
}
\hypertarget{before-we-begin}{%
\section{Before we begin}\label{before-we-begin}}

\hypertarget{literature}{%
\subsection{Literature}\label{literature}}

\begin{itemize}
\tightlist
\item
  {[}HTW{]} Hastie, Tibshirani, Wainwrigh: ``Statistical Learning with
  Sparsity: The Lasso and Generalizations''. CRC press.
  \href{https://hastie.su.domains/StatLearnSparsity/}{Ebook}. Chapter
  4.1-4.3,4.6
\end{itemize}

and for the interested student

\begin{itemize}
\tightlist
\item
  {[}WNvW{]} \href{https://arxiv.org/pdf/1509.09169v7.pdf}{Wessel N. van
  Wieringen: Lecture notes on ridge regression} Chapter 6.6
\item
  {[}CASI{]} Efron and Hastie (2016) Chapter 16 (lasso in general)
\end{itemize}

\begin{center}\rule{0.5\linewidth}{0.5pt}\end{center}

\hypertarget{goal}{%
\subsection{Goal}\label{goal}}

The main goal of this part is to

\begin{itemize}
\tightlist
\item
  know about these special versions of the lasso (also in combination
  with the ridge), and
\item
  to see which practical data situation these can be smart to use.
\end{itemize}

Maybe one of these is suitable for the Data analysis project?

Theoretical properties and algorithmic details are not on the reading
list.

\hypertarget{lasso-and-ridge}{%
\section{Lasso and ridge}\label{lasso-and-ridge}}

We have seen that the ridge regression shrinks the regression
coefficients (as compared to the least squares solution), while the
lasso regression both shrinks and sets some coefficients to zero (model
selection).

\hypertarget{why-do-we-need-lasso-variants}{%
\subsection{Why do we need lasso
variants}\label{why-do-we-need-lasso-variants}}

\begin{center}\rule{0.5\linewidth}{0.5pt}\end{center}

\begin{itemize}
\tightlist
\item
  empirically lasso sometimes does not perform well for highly
  correlated variables
\item
  in some situations we would like a group of covariates to either be in
  or out of the model ``together'', for example a dummy variable coded
  factor
\end{itemize}

\hypertarget{the-l_q-regression}{%
\section{\texorpdfstring{The \(l_q\)
regression}{The l\_q regression}}\label{the-l_q-regression}}

\[ \text{minimize}_{\beta_0,\beta} \{ \sum_{i=1}^N (y_i-\beta_0-\sum_{j=1}^p x_{ij}\beta_j )^2 + \lambda \sum_{j=1}^p \lvert \beta_j\rvert^q \}, q\ge 0\]

\begin{itemize}
\tightlist
\item
  \(q=0\): best subset regression
\item
  \(q=1\): lasso
\item
  \(q=2\): ridge
\end{itemize}

Remark: \(l_q\) is the but penalty of the form
\(\lvert \beta_j\rvert^q\). Note that \(l_2\) is thus the squared
\(L_2\) norm (squared Euclidean norm).

\begin{center}\rule{0.5\linewidth}{0.5pt}\end{center}

Properties

\begin{itemize}
\tightlist
\item
  \(0 \le q \le 1\): not differentiable
\item
  \(1<q<2\): in between lasso and ridge, but differentiable (and no
  variable selection property)
\item
  \(q\) can be estimated from data, but according to Hastie, Tibshirani,
  and Friedman (2009) this is ``not worth the effort for the extra
  variance incurred''
\end{itemize}

\begin{figure}

{\centering \includegraphics[width=1\textwidth,height=\textheight]{./ESLFig312.jpg}

}

\caption{Figure 3.12 from Hastie, Tibshirani, and Friedman (2009)}

\end{figure}

\begin{center}\rule{0.5\linewidth}{0.5pt}\end{center}

\begin{figure}

{\centering \includegraphics[width=1\textwidth,height=\textheight]{./HTWFig412.png}

}

\caption{Figure 4.12 from Hastie, Tibshirani, and Wainwright (2015)}

\end{figure}

\begin{center}\rule{0.5\linewidth}{0.5pt}\end{center}

\hypertarget{bridge-regression}{%
\subsection{Bridge regression}\label{bridge-regression}}

This is \(l_q\) but with \(q>1\).

\[ \text{minimize}_{\beta_0,\beta} \{ \frac{1}{2}\sum_{i=1}^N (y_i-\beta_0-\sum_{j=1}^p x_{ij}\beta_j )^2 + \lambda \sum_{j=1}^p \lvert \beta_j\rvert^q \}, q\ge 0\]

\hypertarget{elastic-net}{%
\section{Elastic net}\label{elastic-net}}

(HTW 4.2) Origin of method: Zou and Hastie (2005)

\begin{itemize}
\tightlist
\item
  Compromise between the ridge and lasso penalty.
\item
  Lasso gives sparsity but does not handle correlated variables well.
\item
  Ridge handles correlated variables well, but is not sparse.
\end{itemize}

Solution: \emph{elastic net} which handles'' coefficients of correlated
features together (similar values or all zero).

The penalty used is now weighted sum of the ridge and the lasso penalty.

\begin{center}\rule{0.5\linewidth}{0.5pt}\end{center}

\hypertarget{minimization-problem}{%
\subsection{Minimization problem}\label{minimization-problem}}

\[ \text{minimize}_{\beta_,\beta} \{ \sum_{i=1}^N (y_i-\beta_0-\sum_{j=1}^p x_{ij}\beta_j )^2 + \lambda \sum_{j=1}^p (\frac{1}{2}(1-\alpha) \beta_j^2+\alpha \lvert \beta_j\rvert )\}\]

again, \(\lambda \ge 0\) is a complexity (regularization, penalty)
parameter controlling the amount of shrinkage together with
\(\alpha \in [0,1]\) that ``balance'' out the penalty between the
squared and absolute penalty.

\begin{itemize}
\tightlist
\item
  \(\alpha=1\): lasso only
\item
  \(\alpha=0\): ridge only
\end{itemize}

The problem is strictly convex if \(\lambda>0\) and \(\alpha<1\) which
gives uniqueness for the elastic net coefficients for this case.

\begin{center}\rule{0.5\linewidth}{0.5pt}\end{center}

\begin{figure}

{\centering \includegraphics[width=1\textwidth,height=\textheight]{./HTWFig42.png}

}

\caption{Figure 4.2 from Hastie, Tibshirani, and Wainwright (2015)}

\end{figure}

\begin{center}\rule{0.5\linewidth}{0.5pt}\end{center}

What is the elastic net parameter constraint region? Why will this give
a variable selection property?

\begin{figure}

{\centering \includegraphics[width=1\textwidth,height=\textheight]{./WNvWFig611.jpg}

}

\caption{Figure 6.11 from Wieringen (2021)}

\end{figure}

Slightly different parametrization in Wieringen (2021), with
\(\lambda_1=\lambda\alpha\) and \(\lambda_2=\lambda(1-\alpha)\).

\begin{center}\rule{0.5\linewidth}{0.5pt}\end{center}

The figure to the right shows potential problems in selecting the best
hyperparameters. Observe that several combination of the two
hyperparameters are equally good.

This is the reason for the parameterization with \(\alpha\) as a mixing
parameter, where the \(\alpha\) is assumed to be set by the user, while
the \(\lambda\) is found using cross-validation.

However, of cause \(\alpha\) is a tuning parameter and need to be set.
See for example the
\href{https://ntnuopen.ntnu.no/ntnu-xmlui/handle/11250/3026839}{Master
thesis of Lene Omdal Tillerli Chapter 3.5 and 5.3} for different
cross-validation strategies for selecting the two hyperparameters.

\begin{center}\rule{0.5\linewidth}{0.5pt}\end{center}

\hypertarget{parameter-estimation}{%
\subsection{Parameter estimation}\label{parameter-estimation}}

The \emph{glmnet}-R package is constructed around the elastic net. Here
the cyclic coordinate descent algorithm is used, and compared to the
pseudo-algorithm we devised in class in L8, for the step with the update
of \(\beta_j\) the soft-threshold solution is slightly modified to (HTW
Equation 4.4)

\[\hat{\beta}_j=\frac{1}{\sum_{i=1}^N x_{ij}^2+\lambda (1-\alpha)} S_{\lambda \alpha}(\sum_{i=1}^N r_{ij}x_{ij})\]

where (as in L8) the soft thresholding operator is
\(S_{\mu}(z)=sign(z)(\lvert z \rvert -\mu)_{+}\) and the partial
residual (as in L8) is
\(r_{ij}=y_i-\beta_0-\sum_{k\neq j} x_{ik}\hat{\beta}_k\) (in L8 we used
\(\tilde{y}\) and not \(r\)).

\begin{center}\rule{0.5\linewidth}{0.5pt}\end{center}

Why do you think this can be solved in such a similar way as for the
lasso?

There is a ``data augmentation trick'' where we can add \(p\) 0-reponses
with covariates \(\sqrt{\lambda (1-\alpha)} \boldsymbol I_{pp}\) to
perform a ridge regression (Wieringen (2021) 6.8.1).

Details are found in the article in the Journal on Statistical Software
on the glmnet Friedman, Hastie, and Tibshirani (2010).

\begin{center}\rule{0.5\linewidth}{0.5pt}\end{center}

\hypertarget{example}{%
\subsection{Example}\label{example}}

This example is shown in Figure 4.2 in HTW and reproduced with the R
code below.

\begin{Shaded}
\begin{Highlighting}[]
\FunctionTok{set.seed}\NormalTok{(}\DecValTok{8701}\NormalTok{)}
\NormalTok{N}\OtherTok{=}\DecValTok{100}
\NormalTok{z1}\OtherTok{=}\FunctionTok{rnorm}\NormalTok{(N,}\DecValTok{0}\NormalTok{,}\DecValTok{1}\NormalTok{); z2}\OtherTok{=}\FunctionTok{rnorm}\NormalTok{(N,}\DecValTok{0}\NormalTok{,}\DecValTok{1}\NormalTok{)}
\NormalTok{eps}\OtherTok{=}\FunctionTok{rnorm}\NormalTok{(N,}\DecValTok{0}\NormalTok{,}\DecValTok{1}\NormalTok{)}

\NormalTok{y}\OtherTok{=}\DecValTok{3}\SpecialCharTok{*}\NormalTok{z1}\FloatTok{{-}1.5}\SpecialCharTok{*}\NormalTok{z2}\SpecialCharTok{+}\DecValTok{2}\SpecialCharTok{*}\NormalTok{eps}

\NormalTok{add}\OtherTok{=}\FunctionTok{matrix}\NormalTok{(}\FunctionTok{rnorm}\NormalTok{(N}\SpecialCharTok{*}\DecValTok{6}\NormalTok{,}\DecValTok{0}\NormalTok{,}\DecValTok{1}\NormalTok{),}\AttributeTok{ncol=}\DecValTok{6}\NormalTok{)}
\NormalTok{x1}\OtherTok{=}\NormalTok{z1}\SpecialCharTok{+}\NormalTok{add[,}\DecValTok{1}\NormalTok{]}\SpecialCharTok{/}\DecValTok{5}\NormalTok{; x2}\OtherTok{=}\NormalTok{z1}\SpecialCharTok{+}\NormalTok{add[,}\DecValTok{2}\NormalTok{]}\SpecialCharTok{/}\DecValTok{5}\NormalTok{; x3}\OtherTok{=}\NormalTok{z1}\SpecialCharTok{+}\NormalTok{add[,}\DecValTok{3}\NormalTok{]}\SpecialCharTok{/}\DecValTok{5}
\NormalTok{x4}\OtherTok{=}\NormalTok{z2}\SpecialCharTok{+}\NormalTok{add[,}\DecValTok{4}\NormalTok{]}\SpecialCharTok{/}\DecValTok{5}\NormalTok{; x5}\OtherTok{=}\NormalTok{z2}\SpecialCharTok{+}\NormalTok{add[,}\DecValTok{5}\NormalTok{]}\SpecialCharTok{/}\DecValTok{5}\NormalTok{; x6}\OtherTok{=}\NormalTok{z2}\SpecialCharTok{+}\NormalTok{add[,}\DecValTok{6}\NormalTok{]}\SpecialCharTok{/}\DecValTok{5}

\NormalTok{x}\OtherTok{=}\FunctionTok{as.matrix}\NormalTok{(}\FunctionTok{data.frame}\NormalTok{(}\AttributeTok{x1=}\NormalTok{x1,}\AttributeTok{x2=}\NormalTok{x2,}\AttributeTok{x3=}\NormalTok{x3,}\AttributeTok{x4=}\NormalTok{x4,}\AttributeTok{x5=}\NormalTok{x5,}\AttributeTok{x6=}\NormalTok{x6))}
\end{Highlighting}
\end{Shaded}

\begin{center}\rule{0.5\linewidth}{0.5pt}\end{center}

\includegraphics{L9_files/figure-pdf/unnamed-chunk-7-1.pdf}

\begin{center}\rule{0.5\linewidth}{0.5pt}\end{center}

\includegraphics{L9_files/figure-pdf/unnamed-chunk-8-1.pdf}

\begin{center}\rule{0.5\linewidth}{0.5pt}\end{center}

\hypertarget{exam-2020-problem-1-stk-in4300-uio}{%
\subsection{Exam 2020 Problem 1 (STK-IN4300,
UiO)}\label{exam-2020-problem-1-stk-in4300-uio}}

Consider data simulated with the following setting:

\begin{itemize}
\tightlist
\item
  \(\beta_i \sim N(0,2), i=1,\ldots,p\)
\item
  \(X \sim N_p(\boldsymbol 0,\boldsymbol \Sigma)\) where
  (i)\(N_p(\cdot,\cdot)\) denotes a \(p\)-dimensional multivariate
  Gaussian distribution; (ii) \(\boldsymbol{0}\) is a \(p\)-dimensional
  vector of \(0\); (iii) \(\boldsymbol \Sigma\) is a \(p \times p\)
  matrix with diagonal elements equal to \(1\) and all other elements
  equal to \(0.9\);
\item
  \(y = X \beta + \varepsilon\), with
  \(\beta = (\beta_1,\ldots, \beta_p)^T\) and
  \(\varepsilon \sim N(0, 1)\).
\end{itemize}

\textbf{a)} If you were forced to choose between ridge regression and
lasso, which one would you have used to predict y on a test set
generated with the same setting? Why?

\textbf{b)} Would your choice have been the same if you ignored the first
information on \(\beta\) ? Why?

\textbf{c)} Do you think that elastic net could have been a better
choice in the situation of point (b)? Why?

\begin{center}\rule{0.5\linewidth}{0.5pt}\end{center}

\hypertarget{group-discussion-exam-2019-problem-1c-stk-in4300-uio}{%
\subsection{Group discussion: Exam 2019 Problem 1c (STK-IN4300,
UiO)}\label{group-discussion-exam-2019-problem-1c-stk-in4300-uio}}

Briefly explain \emph{elastic net} and \emph{bridge regression} and
explain why despite the corresponding constraints are almost
indistinguishable in Figure 3.13 of Hastie, Tibshirani, and Friedman
(2009), they provide, in general, quite different models.

\begin{figure}

{\centering \includegraphics[width=1\textwidth,height=\textheight]{./ESLFig313.jpg}

}

\caption{Figure 3.13 from Hastie, Tibshirani, and Friedman (2009)}

\end{figure}

\hypertarget{group-lasso}{%
\section{Group lasso}\label{group-lasso}}

(HTW Section 4.3.1)

Now we aim at fixing the following problem with the lasso: if we have a
factor and have used dummy variable coding, then the lasso may only
choose to select some of the dummy variables to be in the model, and the
lasso solution also is dependent on how the dummy variable encoding is
done (choosing different contrasts will produce different solutions).
Other application might have groups of genes in pathways, where those
can be handled together.

The solution is to use a penalty that can be seen kind of intermediate
to \(L_1\) and squared \(L_2\):

\begin{center}\rule{0.5\linewidth}{0.5pt}\end{center}

Assume we have \(J\) groups and \(p_j\) covariates for each group.
Further \(Z_j\in \Re^{p_j}\) is the covariates for group \(j\). Further
let \(\theta_j \in \Re^{p_j}\) be the regression coefficients for group
\(j\).

\[\text{minimize}_{\theta_0,\theta} \{ 
\sum_{i=1}^N (y_i-\theta_0-\sum_{j=1}^J z_{ij}\theta_j )^2 
+ \lambda \sum_{j=1}^J \lvert\lvert \theta_j{\rvert \rvert}_2\}\]

where \(\lvert\lvert \theta_j{\rvert \rvert}_2\) is the Euclidean norm
of the vector \(\theta_j\).

\begin{center}\rule{0.5\linewidth}{0.5pt}\end{center}

\hypertarget{what-does-this-new-unsquared-l_2-penalty-do}{%
\subsection{\texorpdfstring{What does this new (unsquared) \(L_2\)
penalty
do?}{What does this new (unsquared) L\_2 penalty do?}}\label{what-does-this-new-unsquared-l_2-penalty-do}}

\begin{itemize}
\tightlist
\item
  All groups with one variable ends up with lasso \(L_1\) penalty
  because: when \(p_j=1\) then
  \(\lvert\lvert \theta_j{\rvert \rvert}_2=\lvert \theta_j{\rvert}\),
  and thus the \(L_1\) lasso penalty is used for singelton groups.
\item
  All groups with more than two variables end up with the square root of
  the ridge penalty. since the penalty is
  \(\sqrt{\sum_{j \in J}\beta_{j}^2}\) for all elements of this group
  \(J\).
\end{itemize}

\begin{center}\rule{0.5\linewidth}{0.5pt}\end{center}

\begin{figure}

{\centering \includegraphics[width=1\textwidth,height=\textheight]{./HTWFig43.png}

}

\caption{Figure 4.3 from Hastie, Tibshirani, and Wainwright (2015)}

\end{figure}

\begin{center}\rule{0.5\linewidth}{0.5pt}\end{center}

Observe that the penalty is the same for all groups, independent of the
group size- but it is common to also include the group size in the
penalty (HTW does not, WNvW does).

HTW Exercise 4.4: the penalty term ensures that the coefficients in a
group sum to zero given that there is an intercept term in the model.

Comment: some results are for orthogonal design matrices for a group.
But, this will only happen if we have a balanced design, with the same
number of observations for each level of a categorical variable group.
This is very seldom the case in observational data, but in Design of
Experiments this may happen for example in \(2^k\) designs.

\begin{center}\rule{0.5\linewidth}{0.5pt}\end{center}

\hypertarget{parameter-estimation-1}{%
\subsection{Parameter estimation}\label{parameter-estimation-1}}

The coordinate descent algorithm may be modified to a block coordinate
descent version. The step to update \(\hat{\theta}_j\) in the coordinate
descent cyclic algorithm is

\[\hat{\theta}_j=({\boldsymbol Z}_j^T{\boldsymbol Z}_j+
\frac{\lambda}{\lvert\lvert \hat{\theta}_j{\rvert \rvert}_2}{\boldsymbol I})^{-1}{\boldsymbol Z}_j^T{\boldsymbol r}_j\]
where as earlier \({\boldsymbol r}_j\) is a partial residual.

If the designmatrix \({\boldsymbol Z}_j\) is ortogonal this is
simplified to \[\hat{\theta}_j=(1
\frac{\lambda}{\lvert \lvert {\boldsymbol Z}_j^T{\boldsymbol r}_j \rvert \rvert})_{+}{\boldsymbol Z}_j^T{\boldsymbol r}_j\]

For non-ortogonal design matrices iterative methods are used.

\begin{center}\rule{0.5\linewidth}{0.5pt}\end{center}

\hypertarget{sparse-group-lasso}{%
\subsection{Sparse group lasso}\label{sparse-group-lasso}}

(HTW Section 4.3.2, WNvW Section 6.8.3)

The group lasso (with the Euclidean penalty) is now joined by the
\(L_1\) penalty. This is kind of similar to the elastic net now the
squared \(L_2\) penalty is replaced by \(L_2\) penalty.

\[\text{minimize}_{\theta_0,\theta} \{ \sum_{i=1}^N (y_i-\theta_0-\sum_{j=1}^J z_{ij}\theta_j )^2 + \lambda \sum_{j=1}^J [(1-\alpha) \lvert \lvert \theta_j {\rvert \rvert}_2 + \alpha \lvert\lvert \theta_j{\rvert \rvert}_1]\}\]

\begin{center}\rule{0.5\linewidth}{0.5pt}\end{center}

\begin{figure}

{\centering \includegraphics[width=1\textwidth,height=\textheight]{./HTWFig45.png}

}

\caption{Figure 4.5 from Hastie, Tibshirani, and Wainwright (2015)}

\end{figure}

\begin{center}\rule{0.5\linewidth}{0.5pt}\end{center}

\hypertarget{parameter-estimation-2}{%
\subsubsection{Parameter estimation}\label{parameter-estimation-2}}

Again, a version of cyclic block-wise coordinate descent can be used.

The case when \(\boldsymbol Z_j\) is not orthogonal requires more work
than for orthonal versions.

Again, as for the elastic net, tuning the two hyperparametres may have
several values being equally good.

\begin{center}\rule{0.5\linewidth}{0.5pt}\end{center}

\hypertarget{group-discussion-exam-2019-problem-1b-stk-in4300-uio}{%
\subsection{Group discussion: Exam 2019 Problem 1b (STK-IN4300,
UiO)}\label{group-discussion-exam-2019-problem-1b-stk-in4300-uio}}

Consider the following version of the sparse group lasso:

\[\text{minimize}_{\beta_0,\beta} \{ \lvert \lvert y-\beta_0 \overrightarrow{1}-\sum_{l=1}^L X_{l}\beta_l {\rvert \rvert}^2_2 + (1-\alpha)\lambda \sum_{l=1}^L\sqrt{p_l}\lvert \lvert \beta_j {\rvert \rvert}_2 + \alpha \lambda \lvert\lvert \beta{\rvert \rvert}_1\}\]
where \(\overrightarrow{1}\) denotes an \(N\)-dimensional vector of 1s,
\(\lambda \ge0\) and \(0\ge \alpha \ge 1\). Answer the following
questions:

\begin{itemize}
\tightlist
\item
  Why does \(\beta_0\) only appear in the first term?
\item
  What happens when \(\alpha=0\) and \(\alpha=1\), respectively?
\item
  Briefly describe the concept of ``bet on sparsity''.
\end{itemize}

\begin{center}\rule{0.5\linewidth}{0.5pt}\end{center}

\hypertarget{overlap-group-lasso}{%
\subsection{Overlap group lasso}\label{overlap-group-lasso}}

(HTW Section 4.3.3)

This is an extension to allow for a covariate to belong to more than one
group.

The overlap group lasso ``replicates a variable'' in whatever group it
is a member of, and then fits the group lasso to the problem.

The overlap group lasso can be used to ensure that interactions between
covariates are only part of the model if the main effects of the
covariates are in the model. See example HTW 4.3 for details.

\begin{center}\rule{0.5\linewidth}{0.5pt}\end{center}

\begin{figure}

{\centering \includegraphics[width=1\textwidth,height=\textheight]{./HTWFig46.png}

}

\caption{Figure 4.6 from Hastie, Tibshirani, and Wainwright (2015)}

\end{figure}

\begin{center}\rule{0.5\linewidth}{0.5pt}\end{center}

\hypertarget{non-convex-penalties}{%
\section{Non-convex penalties}\label{non-convex-penalties}}

(HTW Section 4.5, WNvW Section 6.9)

We have looked at the \(l^q\) penalty formula in the start of L9:

\[ \text{minimize}_{\beta_0,\beta} \{ \sum_{i=1}^N (y_i-\beta_0+\sum_{j=1}^p x_{ij}\beta_j )^2 + \lambda \sum_{j=1}^p \lvert \beta_j\rvert^q \}, q\le 0\]
Observe that if \(0\le q \le 1\) is non-convex.

For \(l^0\) we aim for best subset selection and need to investigate
\(2^p\) possible models. This is not easy for \(p>40\).

The Smoothly Clipped Absolute Deviation SCAD method is an alternative to
the \(l^q\).

\begin{center}\rule{0.5\linewidth}{0.5pt}\end{center}

\hypertarget{adaptive-lasso}{%
\subsection{Adaptive lasso}\label{adaptive-lasso}}

(HTW Section 4.6, WNvW Section 6.8.4)

Origin: Zou (2006)

The aim is to fit models that are even sparser than the lasso. The
method uses a so-called \emph{pilot estimate} \(\tilde{\beta}\):

\[ \text{minimize}_{\beta_0,\beta} \{ \sum_{i=1}^N (y_i-\beta_0+\sum_{j=1}^p x_{ij}\beta_j )^2 + \lambda \sum_{j=1}^p w_j \lvert \beta_j\rvert \}\]
where \(w_j=1/{\lvert \tilde{\beta}_j}\rvert ^{\nu}\) includes the pilot
estimated, and given this pilot estimate the criterion i convex in
\(\beta\). The value of \(\nu\) makes this an approximation to the
\(l^q\) penalty where \(q=1-\nu\).

\begin{center}\rule{0.5\linewidth}{0.5pt}\end{center}

Since the pilot estimate needs to be found first, this can be seen as a
two-step procedure.

If \(p<N\) then the least squares estimator can be used as the pilot
estimate, and for larger \(p\) the ridge or lasso estimate may be used.

If \(\tilde{\beta_j}=0\) then the penalty of the \(j\)th element of the
coefficient vector is infinity and the adaptive lasso estimate for the
coeffisient will be zero.

\begin{center}\rule{0.5\linewidth}{0.5pt}\end{center}

For the orthogonal design, the adaptive lasso can be written as:

\[\hat{\beta}(\lambda)=sign(\hat{\beta}_{{\text LS},j})(\lvert \hat{\beta}_{{\text LS},j}\rvert -\frac{\lambda}{2 \hat{\beta}_{{\text LS},j}} )_{+}\]
Compare this with the lasso, the difference is the last denominator.

\begin{center}\rule{0.5\linewidth}{0.5pt}\end{center}

Unlike the lasso (according to Zou (2006)) the adaptive lasso is found
to fulfill the \emph{oracle property}.

According to Zou (2006), for an oracle procedure \(\delta\) then
\(\beta(\delta)\) has the following properties:

\begin{itemize}
\tightlist
\item
  It identifies the right (correct) subset model,
  \(\{j: \beta_j \neq0\}={\cal A}\)
\item
  ``Has the optimal estimation rate''
  \(\sqrt{N}(\hat{\beta}(\delta)_{\cal A}-\beta^*_{\cal A})\rightarrow_d N(\boldsymbol 0,\boldsymbol \Sigma)\)
\end{itemize}

If stepwise selection is used to find the active set, it can be trapped
in local minima.

The continuous shrinkage property of the lasso is know to improve the
prediction accuracy of the method (bias-variance trade-off).

The adaptive lasso can be estimated using the LARS algorithm of Efron et
al (2004) (not covered in this course, but presented in Hastie,
Tibshirani, and Friedman (2009) Section 3.4.4).

\begin{center}\rule{0.5\linewidth}{0.5pt}\end{center}

\hypertarget{back-to-forward-stepwise-model-selection}{%
\subsection{Back to forward stepwise model
selection}\label{back-to-forward-stepwise-model-selection}}

If the aim is to minimize the squared loss with the \(l^0\) penalty, the
\emph{forward stepwise model} method for model selection is efficent and
``hard to beat''.

The forward stepwise model selection is a greedy algorithm.

\begin{itemize}
\tightlist
\item
  build a model sequentially by adding one variable at a time.
\item
  At each step the best variable to include in the active set is
  identified and
\item
  then the LS-fit is (re)computed for all the active variables.
\end{itemize}

This is an algorithm and not an optimization problem, and the
theoretical properties of the algorithm ``are less well understood''
(HTW page 86).

\hypertarget{a-never-ending-story}{%
\section{A never ending story?}\label{a-never-ending-story}}

There seems to always be something that can be improved upon, and there
are several lasso variantes that we have not discussed.

Other variants include

\begin{itemize}
\tightlist
\item
  The fused lasso (HTW Section 4.5)
\item
  The random lasso
\end{itemize}

\begin{center}\rule{0.5\linewidth}{0.5pt}\end{center}

\hypertarget{group-discussion}{%
\subsection{Group discussion}\label{group-discussion}}

Choose one of the lasso/ridge variants we have covered in L8-L9 and
write down:

\begin{itemize}
\tightlist
\item
  which variation on the classic lasso penalty is used (write down the
  penalty part of the minimization problem)
\item
  make a drawing of the penalty (comparable to the sphere for ridge and
  the diamond for lasso)
\item
  in which practical data analysis situation is this variation used
  (e.g.~when many correlated variables are present, when the covariates
  have a natural group structure, \ldots)
\item
  how can the parameter estimates be found?
\item
  anything else you found interesting?
\end{itemize}

\hypertarget{solutions}{%
\section{Solutions}\label{solutions}}

\hypertarget{exam-uio-stk-in4300-2019-problem-1bc}{%
\subsection{Exam UiO STK-IN4300 2019 Problem
1bc}\label{exam-uio-stk-in4300-2019-problem-1bc}}

(stolen from the webpages at UiO)

\textbf{b)}

\begin{itemize}
\tightlist
\item
  Because the intercept is excluded from the penalization, as it makes
  no sense to shrink it toward 0.
\item
  One obtains the group lasso and the ordinary lasso, respectively.
\item
  It is preferable to use a procedure which assumes a sparse truth over
  one that does not, because the former performs better if the problem
  is actually sparse, while both procedures tend to perform badly for a
  dense problem.
\end{itemize}

\textbf{c)} Bridge regression is a penalize regression approach which
uses a penalty of the form \(\sum_{j=1}^p \lvert \beta_j\rvert^q\) where
\(p\) is the number of explanatory variables and \(\beta_j\) the
regression coefficients. When \(1 \le q \le 2\), the penalty can be seen
as a compromise between the lasso \(q=1\) and the ridge \(q=2\)
penalties.

Similarly, elastic net is also a compromise between lasso and ridge
regression: in this case, a mixture of \(L_1\) and \(L_2\) penalties is
used, with a hyperparameter controlling the ratio between the two
penalties. The resulting methods is supposed to enclose the advantages
of both penalties (mainly, variable selection and a better handling of
correlation, respectively).

The difference in the models that one obtains by applying bridge
regression and elastic net is related to the form of the constraints:
while similar, that of elastic net has non-differentiable corners that
lead to sparser models, as some regression coefficients are forced to be
exactly 0.

\hypertarget{exam-uio-stk-in4300-2020-problem-1}{%
\subsection{Exam UiO STK-IN4300 2020 Problem
1}\label{exam-uio-stk-in4300-2020-problem-1}}

(stolen from the webpages at UiO)

\textbf{a) } Ridge regression, because it performs better in the case of
many variables with small effect (which can be seen from
\(\beta \sim N(0,2)\)) and if there is a strong correlation among the
variables (\(\rho=0.9\)).

\textbf{b)} Without knowing that there are many variables with small
effect (i.e., we miss the information on the \(\beta\)s), it is safer to
use lasso ``betting on sparsity'': if there are only a few variables
with a large ffect, it may strongly outperform ridge; if the situation
is similar to the one at point (a), it will not perform much worse than
ridge regression.

NOTE: if there was good reasoning behind the choice of ridge regression,
the answer was considered correct. Example of good reasoning: Despite
the fact that I do not know the effect of the variables, which would let
me choose lasso, I prefer to use ridge because I want my model to handle
the correlation in a better way, and, anyway, part of the effect of a
few potential strong variables is shared with all the variables due to
correlation.

\textbf{c)} Yes, it would allow having a sparse model due to the \(L_1\)
penalty, with a better handling of the correlation among variables
thanks to the \(L_2\) penalty.

\hypertarget{resources}{%
\section{Resources}\label{resources}}

\begin{itemize}
\tightlist
\item
  Video from webinar with Trevor Hastie on
  \href{http://youtu.be/BU2gjoLPfDc}{glmnet from 2019}
\item
  See
  \href{https://github.com/mettelang/MA8701V2021/blob/main/Part1/LassoandfriendsBenDunn.pdf}{slides
  from guest lecturer Benjamin Dunn in 2021}
\end{itemize}

\hypertarget{references}{%
\section{References}\label{references}}

\hypertarget{refs}{}
\begin{CSLReferences}{1}{0}
\leavevmode\vadjust pre{\hypertarget{ref-casi}{}}%
Efron, Bradley, and Trevor Hastie. 2016. \emph{Computer Age Statistical
Inference - Algorithms, Evidence, and Data Science}. Cambridge
University Press. \url{https://hastie.su.domains/CASI/}.

\leavevmode\vadjust pre{\hypertarget{ref-JSSglmnet}{}}%
Friedman, Jerome H., Trevor Hastie, and Rob Tibshirani. 2010.
{``Regularization Paths for Generalized Linear Models via Coordinate
Descent.''} \emph{Journal of Statistical Software} 33 (1): 1--22.
\url{https://doi.org/10.18637/jss.v033.i01}.

\leavevmode\vadjust pre{\hypertarget{ref-HTW}{}}%
Hastie, Trevor, Roberg Tibshirani, and Martin Wainwright. 2015.
\emph{Statistical Learning with Sparsity: The Lasso and
Generalizations}. CRC Press.
\url{https://hastie.su.domains/StatLearnSparsity/}.

\leavevmode\vadjust pre{\hypertarget{ref-ESL}{}}%
Hastie, Trevor, Robert Tibshirani, and Jerome Friedman. 2009. \emph{The
Elements of Statistical Learning: Data Mining, Inference, and
Prediction}. Vol. 2. Springer series in statistics New York.
\href{https://hastie.su.domains/ElemStatLearn}{hastie.su.domains/ElemStatLearn}.

\leavevmode\vadjust pre{\hypertarget{ref-WNvW}{}}%
Wieringen, Wessel N. van. 2021. {``Lecture Notes on Ridge Regression.''}
\url{https://arxiv.org/pdf/1509.09169v7.pdf}.

\leavevmode\vadjust pre{\hypertarget{ref-Zou2006}{}}%
Zou, Hui. 2006. {``The Adaptive Lasso and Its Oracle Properties.''}
\emph{Journal of the American Statistical Association} 101 (476):
1418--29. \url{https://doi.org/10.1198/016214506000000735}.

\leavevmode\vadjust pre{\hypertarget{ref-ZouHastie2005}{}}%
Zou, Hui, and Trevor Hastie. 2005. {``Regularization and Variable
Selection via the Elastic Net.''} \emph{Journal of the Royal Statistical
Society. Series B (Statistical Methodology)} 67 (2): 301--20.
\url{http://www.jstor.org/stable/3647580}.

\end{CSLReferences}



\end{document}
